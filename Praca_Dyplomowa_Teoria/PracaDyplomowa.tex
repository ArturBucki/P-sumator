\documentclass[12pt, a4paper, onside, polish]{article}				% Wylad dokumentu
\usepackage[utf8]{inputenc}					% Format, utf8
\usepackage[polish]{babel}
\usepackage[T1]{fontenc}
\usepackage{tocloft}
\usepackage{graphicx}
\usepackage{cmap}


\graphicspath{ {./images/} }
\title{Sekcja i rozdziały}						% Strona tytulowa						
\renewcommand{\cftsecleader}{\cftdotfill{\cftdotsep}}




\begin{document}

\begin{titlepage}
	\begin{center}
	
		\includegraphics[width=4cm, height=4cm]{w_uwb_kolor}
		
		\vspace{2cm}

	
		\Huge
        		\textbf{Uniwersytet w Białymstoku}
        		
        		\vspace{0.5cm}
        		\LARGE
        		Instytut Informatyki
        		
        		\vspace{1cm}
        		\LARGE
        		Aplikacja symulująca działanie sumatora/subtraktora w oparciu o jego cyfrowe układy logiczne 
        		
        		\vspace{1.5cm}
        		\large
        		Artur Bucki
        		
        		\large
        		80212
        		
        		\vspace{0.5cm}
        		
		\begin{flushright}
		Promotor:\\
		DR INŻ. WIESŁAW PÓŁJANOWICZ\\
		\end{flushright}


		\vspace*{\fill}
		\large
		Białystok 2022r
        		
	\end{center}
\end{titlepage}





\tableofcontents
\cleardoublepage


\section{Wstęp}
\cleardoublepage



\section{Organizacja i architektura klasycznego komputera}

\subsection{Arytmetyka w systemach cyfrowych}
\subsubsection{Pozycyjne systemy liczbowe}
\cleardoublepage

\subsection{Układy cyfrowe - bramki logiczne}
\cleardoublepage

\subsection{Procesor}
\cleardoublepage

\subsubsection{Jednostka arytmetyczno-logiczna(ALU)}
\cleardoublepage

\subsubsection{Jednostka sterująca}
\cleardoublepage

\subsubsection{Zespół rejestrów}
\cleardoublepage

\subsection{Pamięć}
\cleardoublepage

\subsection{Urządzenia wejścia/wyjścia}
\cleardoublepage

\subsection{Magistrale systemowe.}
\cleardoublepage





\section{Działanie jednostki arytmetyczno-logicznej ALU}
\subsection{Układ sumatora /subtraktora}
\cleardoublepage

\subsection{Działanie układu ALU Simulator - EE3221 Digital Systems II}
\cleardoublepage

      


\section{Programy symulujące działanie układów cyfrowych w komputerze}
\subsection{Digital Works}
\cleardoublepage


\subsection{Cedar logic Simulator}
\cleardoublepage

\subsection{Win Logi lab}
\cleardoublepage

\subsection{Multimedia Logic}
\cleardoublepage

\subsection{Logisim}
\cleardoublepage



\section{Technologie informatyczne wykorzystywane przy budowie aplikacji symulujących działanie układów cyfrowych}
\subsection{C\#}
\cleardoublepage




\subsection{Adobe Photoshop}
\cleardoublepage







\section{Projekt i realizacja aplikacji symulującej działanie sumatora/subtraktora (Add Sub)}
\subsection{Wymagania funkcjonalne i niefunkcjonalne aplikacji}
\cleardoublepage

\subsection{Diagram przypadków użycia (DPU)}
\cleardoublepage

\subsection{Budowa modułowa aplikacji (Add Sub)}
\cleardoublepage

\subsection{Testowanie aplikacji}
\cleardoublepage



\section{Podręcznik użytkownika aplikacji}
\cleardoublepage




\section{Podsumowanie}
\cleardoublepage
\section{Bibliografia}
\cleardoublepage
\section{Spis rysunnków}
\cleardoublepage


\end{document}
